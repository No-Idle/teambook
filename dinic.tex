\subsection{Алгоритм Диница}
\begin{lstlisting}
#define pb push_back
struct Dinic{
struct edge{
    int to, flow, cap;
};

const static int N = 555; //count of vertices

vector<edge> e;
vector<int> g[N + 7];
int dp[N + 7];
int ptr[N + 7];

void clear(){
    for (int i = 0; i < N + 7; i++) g[i].clear();
    e.clear();
}

void addEdge(int a, int b, int cap){
    g[a].pb(e.size());
    e.pb({b, 0, cap});
    g[b].pb(e.size());
    e.pb({a, 0, 0});
}

int minFlow, start, finish;

bool bfs(){
    for (int i = 0; i < N; i++) dp[i] = -1;
    dp[start] = 0;
    vector<int> st;
    int uk = 0;
    st.pb(start);
    while(uk < st.size()){
        int v = st[uk++];
        for (int to : g[v]){
            auto ed = e[to];
            if (ed.cap - ed.flow >= minFlow && dp[ed.to] == -1){
                dp[ed.to] = dp[v] + 1;
                st.pb(ed.to);
            }
        }
    }
    return dp[finish] != -1;
}

int dfs(int v, int flow){
    if (v == finish) return flow;
    for (; ptr[v] < g[v].size(); ptr[v]++){
        int to = g[v][ptr[v]];
        edge ed = e[to];
        if (ed.cap - ed.flow >= minFlow && dp[ed.to] == dp[v] + 1){
            int add = dfs(ed.to, min(flow, ed.cap - ed.flow));
            if (add){
                e[to].flow += add;
                e[to ^ 1].flow -= add;
                return add;
            }
        }
    }
    return 0;
}

int dinic(int start, int finish){
    Dinic::start = start;
    Dinic::finish = finish;
    int flow = 0;
    for (minFlow = (1 << 30); minFlow; minFlow >>= 1){
        while(bfs()){
            for (int i = 0; i < N; i++) ptr[i] = 0;
            while(int now = dfs(start, (int)2e9 + 7)) flow += now;
        }
    }
    return flow;
}
} dinic;
\end{lstlisting}