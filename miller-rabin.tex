\subsection{Алгоритм Миллера --- Рабина}
\begin{lstlisting}
__int128 one=1;
int po(int a,int b,int p)
{
    int res=1;
    while(b) {if(b & 1) {res=(res*one*a)%p;--b;} else {a=(a*one*a)%p;b>>=1;}} return res;
}
bool chprime(int n) ///miller-rabin
{
    if(n==2) return true;
    if(n<=1 || n%2==0) return false;
    int h=n-1;int d=0;while(h%2==0) {h/=2;++d;}
    for(int a:{2, 3, 5, 7, 11, 13, 17, 19, 23, 29, 31, 37})
        {
        if(a==n) return true;
        int u=po(a,h,n);bool ok=0;
        if(u%n==1) continue;
        for(int c=0;c<d;++c)
            {
            if((u+1)%n==0) {ok=1;break;}
            u=(u*one*u)%n;
        }
        if(!ok) return false;
    }
    return true;
}
\end{lstlisting}