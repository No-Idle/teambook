\section{Настройка CLion}
\begin{enumerate}
    \item В файле \texttt{CMakeLists.txt} дописать строчку \texttt{add\_compile\_definitions(LOCAL)}.
    Нажать появившуюся опцию в правом верхнем углу \texttt{enable auto-reload}.
    \item Вбить шаблон в \texttt{main.cpp}:
    \lstinputlisting{template.cpp}
    Скомпилировать, чтобы проверить отсутствие опечаток.
    \item Запустить терминал (crtl + alt + T)
    \begin{lstlisting}
$ cd workspace/CLionProjects
$ for c in {A..Z}; do cp main.cpp $c.cpp && echo "add_executable($c $c.cpp)" >> CMakeLists.txt; done\end{lstlisting}
\end{enumerate}
Далее отключаем подсветку и форматирование в настройках (ctrl+alt+S)
\begin{itemize}
    \item Editor $\to$ Code Style $\to$ Formatter $\to$ \texttt{Do not format} прописать \texttt{*}
    \item Editor $\to$ Inspections $\to$ C/C++ $\to$ static analysis tools $\to$ CLang-Tidy отключить
    \item Editor $\to$ Inlay Hints $\to$ отключаем всё (достаточно первых трёх --- code vision, parameter names, types).
\end{itemize}
Тёмная тема отключается в Appearance \& Behavior $\to$ Appearance.

Чтобы добавить санитайзеры, надо дописать в CMakeLists.txt \texttt{set(CMAKE\_CXX\_FLAGS \texttt{\char`\"}-fsanitize=address -fsanitize=undefined\texttt{\char`\"})}