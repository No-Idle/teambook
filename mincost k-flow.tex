\subsection{Mincost k-flow}
\subsubsection{Строим граф}
\begin{lstlisting}
struct edge {
  int next, capacity, cost, flow = 0;

  edge() = default;

  edge(int next, int capacity, int cost) : next(next), capacity(capacity), cost(cost) {}

  int rem() const { return capacity - flow; }

  int operator+=(int f) { return flow += f; }

  int operator-=(int f) { return flow -= f; }
};
auto addEdge = [&](auto from, auto next, auto capacity, int cost) {
  g[from].push_back(e.size());
  e.emplace_back(next, capacity, cost);
  g[next].push_back(e.size());
  e.emplace_back(from, 0, -cost);
};
\end{lstlisting}
Если граф ориентированный, то addEdge вызываем один раз.
Если неориентированный, то два, вот так:
\begin{lstlisting}
addEdge(u, v, capacity, cost);
addEdge(v, u, capacity, cost);
\end{lstlisting}
\subsubsection{Запускаем Форда --- Беллмана}
\begin{lstlisting}
vector<ll> phi(n, 0);
auto fordBellman = [&](int s, int t) {
  phi.assign(n, 0);
  for (int iter = 0; iter < n; ++iter) {
    bool changed = false;
    for (int u = 0; u < n; ++u) {
      for (auto index : g[u]) {
        auto edge = e[index];
        if (edge.rem() > 0 && phi[edge.next] > phi[u] + edge.cost) {
          phi[edge.next] = phi[u] + edge.cost;
          changed = true;
        }
      }
    }
    if (!changed) break;
  }
};
fordBellman(s, t);
\end{lstlisting}

\subsubsection{Ищем кратчайший путь Дейкстрой с потенциалами}
\begin{lstlisting}
vector<ll> dist;
vector<int> from;
vector<bool> cnt;
auto dijkstra = [&](int s, int t) {
  dist.assign(n, 1e18);
  from.assign(n, -1);
  cnt.assign(n, false);
  dist[s] = 0;
  for (int i = 1; i < n; ++i) {
    int cur = find(cnt.begin(), cnt.end(), false) - cnt.begin();
    for (int j = 0; j < n; ++j) {
      if (!cnt[j] && dist[j] < dist[cur]) cur = j;
    }
    cnt[cur] = true;
    for (int index : g[cur]) {
      auto &edge = e[index];
      if (edge.rem() == 0) continue;
      ll weight = edge.cost + phi[cur] - phi[edge.next];
      if (dist[edge.next] > dist[cur] + weight) {
        dist[edge.next] = dist[cur] + weight;
        from[edge.next] = cur;
      }
    }
  }
  if (dist[t] == (ll) 1e18) return -1LL;
  ll cost = 0;
  for (int p = t; p != s; p = from[p]) {
    for (auto index : g[from[p]]) {
      auto &edge = e[index];
      ll weight = edge.cost + phi[from[p]] - phi[edge.next];
      if (edge.rem() > 0 && edge.next == p && dist[edge.next] == dist[from[p]] + weight) {
        edge += 1;
        e[index ^ 1] -= 1;
        cost += edge.cost;
        break;
      }
    }
  }
  for (int i = 0; i < n; ++i) {
    phi[i] += dist[i];
  }
  return cost;
};
ll cost = 0;
for (int flow = 0; flow < k; ++flow) {
  ll a = dijkstra(s, t);
  if (a == -1) {
    cout << "-1\n";
    return;
  }
  cost += a;
}
\end{lstlisting}
\subsubsection{Восстанавливаем ответ}
\begin{lstlisting}
auto findPath = [&](int s, int t) {
  vector<int> ans;
  int cur = s;
  while (cur != t) {
    for (auto index : g[cur]) {
      auto &edge = e[index];
      if (edge.flow <= 0) continue;
      edge -= 1;
      e[index ^ 1] += 1;
      ans.push_back(index / 4);
// index / 4 because each edge has 4 copies
      cur = edge.next;
      break;
    }
  }
  return ans;
};
for (int flow = 0; flow < k; ++flow) {
  auto p = findPath(s, t);
  cout << p.size() << ' ';
  for (int x : p) cout << x + 1 << ' ';
  cout << '\n';
}
\end{lstlisting}