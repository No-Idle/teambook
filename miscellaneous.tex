\subsection{Нимы}
По умолчанию проигрывает тот, кто не может сделать ход.
\begin{enumerate}
\item Полоска $1\times n$. Надо ставить крестик в незанятую клетку, нельзя ставить в соседние.
\textbf{Решение:} динамика за $\mathcal{O}(n^2)$. Оказывается, начиная с $n = 52$ есть период длины $34$.
\item Полоска $1\times n$. Надо ставить крестик в незанятую клетку. Выигрывает тот, кто получит три крестика подряд. \textbf{Решение:} как в (1), только крестик банит слева и справа по две клетки.
\item Доска $3 \times n$, изначально стоят $n$ белых пешек в первом ряду, $n$ чёрных в последнем. Правила шахматные, но бить обязательно. \textbf{Решение: } совпадает с (1).
\item Ним, но ещё за ход разрешается вместо взятия камней разделить какую-то кучку на две. \textbf{Решение:} динамика за квадрат. Оказывается, $g[n] = n + s_n$, где $s_n = 1$ для $n \equiv 3 \pmod{4}$, $s_n = -1$ для $n \equiv 0 \pmod {4}$, $s_n = 0$ в остальных случаях (и, конечно, при $n = 0$).
\item Полоска $1 \times n$. Надо поставить крестик в незанятую клетку, или пару крестиков в соседние незанятые клетки. \textbf{Решение:} динамика за квадрат. Оказывается, начиная с некоторого места $(n = 60?)$ последовательность периодична с периодом $12$.
\item $n$ кучек камней. Надо разделить кучку размера $\ge 3$ на две неравные. \textbf{Решение:} динамика за квадрат. Период неизвестен человечеству.
\item $n$ кучек камней. Надо переместить ненулевое число камней из $i$-й кучки в $(i-1)$-ю (для $i > 1$). \textbf{Решение:} нимбер --- это $a_2 \oplus a_4 \oplus a_6 \oplus \dots$, так как это ним с увеличениями.
\item Полоска $1 \times n$, на ней $k$ фишек. Надо переместить любую фишку куда-то влево, не перепрыгивая и не вставая на другие фишки. \textbf{Решение:} сводимся к предыдущей задаче
\item Полоска $1 \times n$, в каждой клетке $o$ илм $x$. Надо поменять $o$ на $x$, после чего (разрешается|надо) флипнуть знак где-то слева. \textbf{Решение:} нимбер --- $\oplus$-сумма координат нулей (в $1$|в $0$)-индексации.
\item Полоска $1 \times n$, в ней стоят две фишки $I$ и $I\:I$. Игрок берёт свою фишку, и перемещает на ненулевое число клеток, не перепрыгивая и не вставая на фишку противника. \textbf{Решение:} нимбер --- число клеток между фишками.
\item Доска $2 \times n$. Надо поставить крестики в $3$ незанятые клетки вида триомино (связная по сторонам фигура из трёх клеток, не лежащих на одной прямой). \textbf{Решение:} динамика за квадрат --- фигуру площади $n$ всегда можно разбить на $k$ и $n-3-k$.
\item Ним-Баше: можно брать не более $k$ предметов. \textbf{Решение: } эквивалентно ниму с состояниями $a_i \pmod{k+1}$.
\item Ним Мура: можно брать сколько угодно из не более, чем $k$ кучек. \textbf{Решение:} проигрышная тогда и только тогда если взять все кучки, записанные в двоичной системе счисления, то в каждом бите сумма делится на $(k+1)$.
\item Ним в поддавки: \textbf{Решение:} нимбер --- $a_1 \oplus \ldots \oplus a_n \oplus [\text{все }a_i \le 1]$.
\end{enumerate}
\subsection{Компараторы}
\lstinputlisting{algos/comparators guide.cpp}
\subsection{Трюки от Сергея Копелиовича}
\subsubsection{Быстрый ввод}
\underline{\url{https://acm.math.spbu.ru/~sk1/algo/input-output}}
\lstinputlisting{algos/fast-input.cpp}
\lstinputlisting{algos/read_double.cpp}
\underline{\url{https://acm.math.spbu.ru/~sk1/algo/memory.cpp.html}}
\subsubsection{Быстрый аллокатор}
\lstinputlisting{algos/bump-alloc.cpp}
\subsection{Редукция Барретта}
\lstinputlisting{algos/barrett.cpp}
\subsection{Флаги компияции}
\texttt{-DLOCAL -Wall -Wextra -pedantic -Wshadow -Wformat=2 -Wfloat-equal -Wconversion -Wlogical-op -Wshift-overflow=2 -Wduplicated-cond -Wcast-qual -Wcast-align -D\_GLIBCXX\_DEBUG -D\_GLIBCXX\_DEBUG\_PEDANTIC -D\_FORTIFY\_SOURCE=2 -fsanitize=address -fsanitize=undefined -fno-sanitize-recover -fstack-protector -std=c++2a}
%\subsubsection{Сеточка в vim}
%\underline{\url{https://codeforces.com/blog/entry/122540}}
%\begin{lstlisting}
%i|<esc>25A   |<esc>
%o+<esc>25A---+<esc>
%Vky35Pdd
%\end{lstlisting}
\subsection{Что сделать на пробном туре}
\begin{itemize}
\item Послать клар
\item Распечатать что-то
\item Получить ML (stack \& heap)
\item Максимальный размер отправляемого файла?
\item Убедиться, что чекер регистронезависимый (yes/YES)
\item Позапускать Флойда --- Варшалла
\item Посмотреть, насколько быстр быстрый ввод
\item Перебить что-то, проверить хеш
\item Проверить санитайзеры
\end{itemize}
\subsection{Хеш файла без комментариев}
Хеш файла, игнорирующий переводы строк и комментарии:
\begin{lstlisting}
$ cpp -dD -P -fpreprocessed "$filename" | tr -d '[:space:]' | md5sum | cut -c-6
\end{lstlisting}
