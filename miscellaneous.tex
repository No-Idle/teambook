\subsection{Компараторы}
\lstinputlisting{algos/comparators guide.cpp}
\subsection{Трюки от Сергея Копелиовича}
\subsubsection{Быстрый ввод}
\lstinputlisting{algos/fast-input.cpp}
\subsubsection{Быстрый аллокатор}
\lstinputlisting{algos/bump-alloc.cpp}
\subsection{Шаблон}
\lstinputlisting{template.cpp}
\subsection{Флаги компияции}
\texttt{-DLOCAL -Wall -Wextra -pedantic -Wshadow -Wformat=2 -Wfloat-equal -Wconversion -Wlogical-op -Wshift-overflow=2 -Wduplicated-cond -Wcast-qual -Wcast-align -D\_GLIBCXX\_DEBUG -D\_GLIBCXX\_DEBUG\_PEDANTIC -D\_FORTIFY\_SOURCE=2 -fsanitize=address -fsanitize=undefined -fno-sanitize-recover -fstack-protector -std=c++2a}
\subsubsection{Сеточка в vim}
\underline{\url{https://codeforces.com/blog/entry/122540}}

\begin{lstlisting}
i|<esc>25A   |<esc>
o+<esc>25A---+<esc>
Vky35Pdd
\end{lstlisting}
\subsection{Что сделать на пробном туре}
\begin{itemize}
\item Убедиться, что работаеют все IDE.
Разобраться, как настраивать в них LOCAL.
\item В системе ML --- это ML или RE?
\item Максимальный размер файла
\item Можно посмотреть на время работы серверов позапускав Флойда --- Варшалла
\end{itemize}
