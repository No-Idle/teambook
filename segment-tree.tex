\subsection{Дерево отрезков}
\begin{lstlisting}
template<typename Data, typename Mod, typename UniteData, typename UniteMod, typename Apply>
struct MassSegmentTree {
    int h, n;
    Data zd;
    Mod zm;
    vector<Data> data;
    vector<Mod> mod;

    UniteData ud; // Data (Data, Data)
    UniteMod um; // Mod (Mod, Mod);
    Apply a; // Data (Data, Mod, int); last argument is the length of current segment (could be used for range += and sum counting, for instance)

    template<typename I>
    MassSegmentTree(int sz, Data zd, Mod zm, UniteData ud, UniteMod um, Apply a, I init) : h(__lg(sz > 1 ? sz - 1 : 1) + 1), n(1 << h), zm(zm), zd(zd), data(2 * n, zd), mod(n, zm), ud(ud), um(um), a(a) {
        for (int i = 0; i < sz; ++i) data[i + n] = init(i);
        for (int i = n - 1; i > 0; --i) data[i] = ud(data[2 * i], data[2 * i + 1]);
    }

    MassSegmentTree(int sz, Data zd, Mod zm, UniteData ud, UniteMod um, Apply a) : h(__lg(sz > 1 ? sz - 1 : 1) + 1), n(1 << h), zm(zm), zd(zd), data(2 * n, zd), mod(n, zm), ud(ud), um(um), a(a) {}

    void push(int i) {
        if (mod[i] == zm) return;
        apply(2 * i, mod[i]);
        apply(2 * i + 1, mod[i]);
        mod[i] = zm;
    }

    // is used only for apply
    int length(int i) { return 1 << (h - __lg(i)); }

    // is used only for descent
    int left(int i) {
        int lvl = __lg(i);
        return (i & ((1 << lvl) - 1)) * (1 << (h - lvl));
    }

    // is used only for descent
    int right(int i) {
        int lvl = __lg(i);
        return ((i & ((1 << lvl) - 1)) + 1) * (1 << (h - lvl));
    }

    template<typename S>
    void apply(int i, S x) {
        data[i] = a(data[i], x, length(i));
        if (i < n) mod[i] = um(mod[i], x);
    }

    void update(int i) {
        if (mod[i] != zm) return;
        data[i] = ud(data[2 * i], data[2 * i + 1]);
    }

    template<typename S>
    void update(int l, int r, S x) { // [l; r)
        l += n, r += n;
        for (int shift = h; shift > 0; --shift) {
            push(l >> shift);
            push((r - 1) >> shift);
        }
        for (int lf = l, rg = r; lf < rg; lf /= 2, rg /= 2) {
            if (lf & 1) apply(lf++, x);
            if (rg & 1) apply(--rg, x);
        }
        for (int shift = 1; shift <= h; ++shift) {
            update(l >> shift);
            update((r - 1) >> shift);
        }
    }

    Data get(int l, int r) { // [l; r)
        l += n, r += n;
        for (int shift = h; shift > 0; --shift) {
            push(l >> shift);
            push((r - 1) >> shift);
        }
        Data leftRes = zd, rightRes = zd;
        for (; l < r; l /= 2, r /= 2) {
            if (l & 1) leftRes = ud(leftRes, data[l++]);
            if (r & 1) rightRes = ud(data[--r], rightRes);
        }
        return ud(leftRes, rightRes);
    }

    // l \in [0; n) && ok(get(l, l), l);
    // returns last r: ok(get(l, r), r)
    template<typename C>
    int lastTrue(int l, C ok) {
        l += n;
        for (int shift = h; shift > 0; --shift) push(l >> shift);
        Data cur = zd;
        do {
            l >>= __builtin_ctz(l);
            Data with1;
            with1 = ud(cur, data[l]);
            if (ok(with1, right(l))) {
                cur = with1;
                ++l;
            } else {
                while (l < n) {
                    push(l);
                    Data with2;
                    with2 = ud(cur, data[2 * l]);
                    if (ok(with2, right(2 * l))) {
                        cur = with2;
                        l = 2 * l + 1;
                    } else {
                        l = 2 * l;
                    }
                }
                return l - n;
            }
        } while (l & (l - 1));
        return n;
    }

    // r \in [0; n) && ok(get(r, r), r);
    // returns first l: ok(get(l, r), l)
    template<typename C>
    int firstTrue(int r, C ok) {
        r += n;
        for (int shift = h; shift > 0; --shift) push((r - 1) >> shift);
        Data cur = zd;
        while (r & (r - 1)) {
            r >>= __builtin_ctz(r);
            Data with1;
            with1 = ud(data[--r], cur);
            if (ok(with1, left(r))) {
                cur = with1;
            } else {
                while (r < n) {
                    push(r);
                    Data with2;
                    with2 = ud(data[2 * r + 1], cur);
                    if (ok(with2, right(2 * r))) {
                        cur = with2;
                        r = 2 * r;
                    } else {
                        r = 2 * r + 1;
                    }
                }
                return r - n + 1;
            }
        }
        return 0;
    }
};
\end{lstlisting}
\subsubsection{Примеры:}
\begin{itemize}
    \item Взятие максимума и прибавление константы
\begin{lstlisting}
MassSegmentTree segtree(n, 0LL, 0LL,
[](int x, int y) { return max(x, y); },
[](int x, int y) { return x + y; },
[](int x, int y, int len) { return x + y; });
\end{lstlisting}

    \item Взятие суммы и прибавление константы
\begin{lstlisting}
MassSegmentTree segtree(n, 0LL, 0LL,
[](int x, int y) { return x + y; },
[](int x, int y) { return x + y; },
[](int x, int y, int len) { return x + y * len; });
\end{lstlisting}

\item Взятие суммы и присовение
\begin{lstlisting}
MassSegmentTree segtree(n, 0LL, -1LL,
[](int x, int y) { return x + y; },
[](int x, int y) { return y; },
[](int x, int y, int len) { return y * len; });
\end{lstlisting}
\end{itemize}